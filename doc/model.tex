\documentclass[a4paper]{article}
\usepackage{fullpage}
\usepackage{epsfig}
\usepackage{pdfsync} 
\usepackage{amsfonts}
\usepackage{amsmath} 
\begin{document}

\title{Notes on the Duffy model}
\author{Anand Patil}
\maketitle

\section{License} % (fold)
\label{sec:license}
This document is licensed under the Creative Commons attribution share-alike license, see \texttt{LICENSE} in the root directory. Copyright \copyright\ 2009 Anand Patil
% section license (end)

\section{Likelihood} % (fold)
\label{sec:likelihood}

The a/b switch mutation happens with probability $P(b=1)=p_{ab}(x)$, which should be much higher for $x$ in Africa. Given that $b=1$, the silencing mutation in the promoter region happens with probability $p_0(x)$. Given that $b=0$, the silencing mutation happens with probability $p_1$, which is assumed to be a small, constant value. Hardy-Weinberg assumptions apply to the genotype frequencies.
\begin{description}
    % \item[lon,lat]: Standard 
    % \item[n]: Sample size
    % \item[africa]: Whether the point was taken in Africa
    % \item[data]: The type of data
    \item[gen*]: Genotype data. The haplotype frequencies are:
    \begin{description}
        \item[gena] $(1-p_{ab}(x))(1-p_1)$
        \item[genb] $p_{ab}(x)(1-p_0(x))$ 
        \item[gen0] $p_{ab}(x)p_0(x)$
        \item[gen1] $(1-p_{ab}(x))$
    \end{description}
    The genotype frequencies (genaa, genab, etc.) can be obtained using the standard Hardy-Weinberg formulas. For example, the frequency of $genab$ is twice the product of the frequencies of $gena$ and $genb$, which is $2(1-p_{ab}(x))(1-p_1)p_{ab}(x)(1-p_0(x))$
    \item[phe*]: Phenotype data.
    \begin{description}
        \item[pheab] This can only happen if the genotype is genab.
        \item[phea] This can happen if the genotype is gena0, gena1 or genaa.
        \item[pheb] This can happen if the genotype is genb0, genb1 or genbb.
        \item[phe0] This can only happen if the phenotype is gen00, gen01 or gen11.
    \end{description}
    \item[aphe*]: Phenotype data, Fya+/- only.
    \begin{description}
        \item[aphea] This can happen if the genotype is genaa, genab, gena1 or gena0.
        \item[aphe0] The complement of aphea.
    \end{description}
    \item[bphe*] Phenotype data, Fyb+/- only.
    \begin{description}
        \item[bpheb] This can happen if the genotype is genbb, genab, genb0 or genb1.
        \item[bphe0] The complement of bpheb. 
    \end{description}
    \item[prom]: Molecular study looked only at promoter region.
    \begin{description}
        \item[prom0]: This can only happen if the genotype is gen00, gen01 or gen11.
        \item[promab]: The complement of pos0.
    \end{description}
\end{description}


\end{document}